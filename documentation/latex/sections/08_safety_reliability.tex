\section{Safety and Reliability Analysis}
\label{sec:safety_reliability}

This section provides a comprehensive analysis of the safety mechanisms and reliability features implemented in the Robotic Ultrasound System, ensuring patient safety and system dependability in clinical environments.

\subsection{Safety Requirements and Standards}
\label{subsec:safety_requirements}

The system adheres to international medical device standards and implements multiple layers of safety mechanisms to ensure patient protection and operational safety.

\subsubsection{Applicable Standards Compliance}

\begin{table}[h]
\centering
\begin{tabular}{|l|p{6cm}|c|}
\hline
\textbf{Standard} & \textbf{Description} & \textbf{Compliance Level} \\
\hline
IEC 60601-1 & Medical electrical equipment - General requirements for basic safety and essential performance & Full \\
\hline
IEC 62304 & Medical device software - Software life cycle processes & Full \\
\hline
ISO 13485 & Quality management systems for medical devices & Full \\
\hline
IEC 62366-1 & Application of usability engineering to medical devices & Partial \\
\hline
ISO 14971 & Application of risk management to medical devices & Full \\
\hline
FDA 21 CFR 820 & Quality System Regulation for medical devices & Full \\
\hline
\end{tabular}
\caption{Medical Device Standards Compliance}
\label{tab:standards_compliance}
\end{table}

\subsubsection{Risk Assessment Framework}

The system implements a comprehensive risk assessment following ISO 14971:

\begin{figure}[h]
\centering
\begin{tikzpicture}[scale=1.1]
% Risk assessment process flow
\node[draw, rectangle, rounded corners, minimum width=3cm, minimum height=1cm] (identify) at (0,6) {Risk Identification};
\node[draw, rectangle, rounded corners, minimum width=3cm, minimum height=1cm] (analyze) at (0,4.5) {Risk Analysis};
\node[draw, rectangle, rounded corners, minimum width=3cm, minimum height=1cm] (evaluate) at (0,3) {Risk Evaluation};
\node[draw, diamond, minimum size=1.5cm] (acceptable) at (0,1.5) {Acceptable?};
\node[draw, rectangle, rounded corners, minimum width=3cm, minimum height=1cm] (control) at (4,1.5) {Risk Control};
\node[draw, rectangle, rounded corners, minimum width=3cm, minimum height=1cm] (validate) at (4,3) {Validation};
\node[draw, rectangle, rounded corners, minimum width=3cm, minimum height=1cm] (monitor) at (0,0) {Monitor};

% Arrows
\draw[->] (identify) -- (analyze);
\draw[->] (analyze) -- (evaluate);
\draw[->] (evaluate) -- (acceptable);
\draw[->] (acceptable) -- (control) node[midway, above] {No};
\draw[->] (acceptable) -- (monitor) node[midway, right] {Yes};
\draw[->] (control) -- (validate);
\draw[->] (validate) -- (evaluate);

\end{tikzpicture}
\caption{Risk Management Process Flow}
\label{fig:risk_management_flow}
\end{figure}

\subsection{Safety Mechanisms Implementation}
\label{subsec:safety_mechanisms}

The system incorporates multiple layers of safety mechanisms to prevent hazardous situations and ensure safe operation.

\subsubsection{Emergency Stop System}

\begin{lstlisting}[language=C++, caption=Emergency Stop Implementation]
class EmergencyStopSystem {
public:
    enum class StopReason {
        USER_INITIATED,
        COLLISION_DETECTED,
        FORCE_LIMIT_EXCEEDED,
        COMMUNICATION_LOST,
        SYSTEM_FAULT,
        WORKSPACE_VIOLATION
    };
    
private:
    std::atomic<bool> emergencyStopActive_{false};
    std::vector<std::function<void()>> emergencyCallbacks_;
    std::mutex callbackMutex_;
    
    // Hardware emergency stop monitoring
    std::thread emergencyMonitorThread_;
    std::atomic<bool> monitoringActive_{true};
    
    // Safety-rated hardware interfaces
    SafetyIO safetyIO_;
    WatchdogTimer watchdog_;
    
public:
    EmergencyStopSystem() {
        // Initialize safety-rated hardware
        safetyIO_.initialize();
        watchdog_.initialize(std::chrono::milliseconds(100)); // 100ms watchdog
        
        // Start emergency monitoring thread
        emergencyMonitorThread_ = std::thread(&EmergencyStopSystem::monitorEmergencyInputs, this);
        
        // Register with system-wide safety monitor
        SystemSafetyMonitor::instance().registerEmergencyStop(this);
    }
    
    ~EmergencyStopSystem() {
        monitoringActive_ = false;
        if (emergencyMonitorThread_.joinable()) {
            emergencyMonitorThread_.join();
        }
    }
    
    void triggerEmergencyStop(StopReason reason, const std::string& details = "") {
        if (emergencyStopActive_.exchange(true)) {
            return; // Already in emergency stop
        }
        
        log(CRITICAL, "EMERGENCY STOP TRIGGERED: " + stopReasonToString(reason) + 
            (details.empty() ? "" : " - " + details));
        
        // Immediate hardware safety actions
        executeHardwareEmergencyStop();
        
        // Notify all registered callbacks
        {
            std::lock_guard<std::mutex> lock(callbackMutex_);
            for (const auto& callback : emergencyCallbacks_) {
                try {
                    callback();
                } catch (const std::exception& e) {
                    log(ERROR, "Emergency callback failed: " + std::string(e.what()));
                }
            }
        }
        
        // Record emergency stop event
        recordEmergencyEvent(reason, details);
        
        // Notify safety monitoring system
        SystemSafetyMonitor::instance().notifyEmergencyStop(reason, details);
    }
    
    void registerEmergencyCallback(std::function<void()> callback) {
        std::lock_guard<std::mutex> lock(callbackMutex_);
        emergencyCallbacks_.push_back(std::move(callback));
    }
    
    bool isEmergencyStopActive() const {
        return emergencyStopActive_.load(std::memory_order_acquire);
    }
    
    void resetEmergencyStop() {
        if (!emergencyStopActive_) {
            return;
        }
        
        // Verify system is safe to reset
        if (!performSafetyChecks()) {
            log(ERROR, "Cannot reset emergency stop: safety checks failed");
            return;
        }
        
        // Reset hardware safety systems
        safetyIO_.reset();
        watchdog_.reset();
        
        emergencyStopActive_.store(false, std::memory_order_release);
        log(INFO, "Emergency stop reset - system ready");
    }
    
private:
    void monitorEmergencyInputs() {
        while (monitoringActive_) {
            // Check hardware emergency stop buttons
            if (safetyIO_.isEmergencyStopPressed()) {
                triggerEmergencyStop(StopReason::USER_INITIATED, "Hardware E-Stop");
            }
            
            // Check communication watchdog
            if (watchdog_.hasExpired()) {
                triggerEmergencyStop(StopReason::COMMUNICATION_LOST, "Watchdog timeout");
            }
            
            // Check force sensors
            if (checkForceLimits()) {
                triggerEmergencyStop(StopReason::FORCE_LIMIT_EXCEEDED, "Force threshold exceeded");
            }
            
            std::this_thread::sleep_for(std::chrono::milliseconds(10));
        }
    }
    
    void executeHardwareEmergencyStop() {
        // Immediately stop all actuators
        safetyIO_.disableAllActuators();
        
        // Activate electromagnetic brakes
        safetyIO_.activateBrakes();
        
        // Cut power to non-essential systems
        safetyIO_.cutNonEssentialPower();
        
        // Signal external safety systems
        safetyIO_.activateSafetyBeacon();
    }
    
    bool performSafetyChecks() {
        // Verify all emergency conditions are cleared
        if (safetyIO_.isEmergencyStopPressed()) return false;
        if (!safetyIO_.allSystemsNominal()) return false;
        if (!checkForceLimits()) return false;
        
        // Verify operator acknowledgment
        if (!SystemSafetyMonitor::instance().hasOperatorAcknowledgment()) return false;
        
        return true;
    }
    
    bool checkForceLimits() {
        const auto forces = safetyIO_.readForceSensors();
        const double maxAllowedForce = 50.0; // Newton
        
        for (const auto& force : forces) {
            if (force.magnitude() > maxAllowedForce) {
                return false;
            }
        }
        return true;
    }
};
\end{lstlisting}

\subsubsection{Collision Detection and Avoidance}

\begin{lstlisting}[language=C++, caption=Advanced Collision Detection System]
class CollisionDetectionSystem {
private:
    struct CollisionGeometry {
        std::vector<ConvexHull> robotLinks;
        std::vector<ConvexHull> obstacles;
        std::vector<Sphere> safetyZones;
        AABB worldBounds;
    };
    
    CollisionGeometry geometry_;
    std::shared_ptr<BVHTree> bvhTree_;
    
    // Real-time collision monitoring
    std::thread collisionMonitorThread_;
    std::atomic<bool> monitoringActive_{true};
    std::chrono::milliseconds checkInterval_{5}; // 5ms = 200Hz
    
    // Safety distances
    static constexpr double CRITICAL_DISTANCE = 0.05; // 5cm
    static constexpr double WARNING_DISTANCE = 0.15;  // 15cm
    static constexpr double SAFETY_MARGIN = 0.02;     // 2cm
    
public:
    enum class CollisionRisk {
        NONE,
        WARNING,
        CRITICAL,
        IMMINENT
    };
    
    CollisionDetectionSystem() {
        initializeGeometry();
        buildBVHTree();
        
        // Start real-time monitoring
        collisionMonitorThread_ = std::thread(&CollisionDetectionSystem::monitorCollisions, this);
    }
    
    CollisionRisk checkCollisionRisk(const RobotState& currentState,
                                   const RobotState& predictedState,
                                   double timeHorizon) {
        
        // Update robot geometry to current state
        updateRobotGeometry(currentState);
        
        // Check current state for collisions
        auto currentRisk = checkInstantaneousCollision();
        if (currentRisk >= CollisionRisk::CRITICAL) {
            return currentRisk;
        }
        
        // Check predicted trajectory for future collisions
        auto trajectoryRisk = checkTrajectoryCollision(currentState, predictedState, timeHorizon);
        
        return std::max(currentRisk, trajectoryRisk);
    }
    
    bool validateTrajectory(const Trajectory& trajectory) {
        for (size_t i = 0; i < trajectory.size(); ++i) {
            updateRobotGeometry(trajectory[i].state);
            
            if (checkInstantaneousCollision() >= CollisionRisk::WARNING) {
                log(WARNING, "Trajectory collision detected at waypoint " + std::to_string(i));
                return false;
            }
        }
        return true;
    }
    
    std::vector<Vector3d> getCollisionPoints() const {
        std::vector<Vector3d> collisionPoints;
        
        // Check all robot link pairs against obstacles
        for (const auto& robotLink : geometry_.robotLinks) {
            for (const auto& obstacle : geometry_.obstacles) {
                auto contact = computeClosestPoints(robotLink, obstacle);
                if (contact.distance < CRITICAL_DISTANCE) {
                    collisionPoints.push_back(contact.point);
                }
            }
        }
        
        return collisionPoints;
    }
    
private:
    CollisionRisk checkInstantaneousCollision() {
        double minDistance = std::numeric_limits<double>::max();
        
        // Use BVH tree for efficient collision queries
        for (const auto& robotLink : geometry_.robotLinks) {
            auto nearbyObstacles = bvhTree_->query(robotLink.getBoundingBox());
            
            for (const auto& obstacle : nearbyObstacles) {
                double distance = computeDistance(robotLink, *obstacle);
                minDistance = std::min(minDistance, distance);
                
                if (distance < SAFETY_MARGIN) {
                    return CollisionRisk::IMMINENT;
                }
            }
        }
        
        if (minDistance < CRITICAL_DISTANCE) return CollisionRisk::CRITICAL;
        if (minDistance < WARNING_DISTANCE) return CollisionRisk::WARNING;
        return CollisionRisk::NONE;
    }
    
    CollisionRisk checkTrajectoryCollision(const RobotState& current,
                                         const RobotState& predicted,
                                         double timeHorizon) {
        
        const int numSteps = static_cast<int>(timeHorizon / 0.01); // 10ms steps
        CollisionRisk maxRisk = CollisionRisk::NONE;
        
        for (int step = 1; step <= numSteps; ++step) {
            double t = static_cast<double>(step) / numSteps;
            RobotState interpolatedState = interpolate(current, predicted, t);
            
            updateRobotGeometry(interpolatedState);
            CollisionRisk stepRisk = checkInstantaneousCollision();
            
            maxRisk = std::max(maxRisk, stepRisk);
            
            if (maxRisk >= CollisionRisk::CRITICAL) {
                break; // Early termination for critical collisions
            }
        }
        
        return maxRisk;
    }
    
    void monitorCollisions() {
        while (monitoringActive_) {
            auto startTime = std::chrono::high_resolution_clock::now();
            
            // Get current robot state
            auto currentState = SystemStateManager::instance().getCurrentRobotState();
            auto predictedState = SystemStateManager::instance().getPredictedRobotState(0.1); // 100ms ahead
            
            // Check collision risk
            CollisionRisk risk = checkCollisionRisk(currentState, predictedState, 0.1);
            
            // Take appropriate action based on risk level
            switch (risk) {
                case CollisionRisk::IMMINENT:
                    EmergencyStopSystem::instance().triggerEmergencyStop(
                        EmergencyStopSystem::StopReason::COLLISION_DETECTED,
                        "Imminent collision detected");
                    break;
                    
                case CollisionRisk::CRITICAL:
                    MotionController::instance().activateEmergencyBraking();
                    log(CRITICAL, "Critical collision risk - emergency braking activated");
                    break;
                    
                case CollisionRisk::WARNING:
                    MotionController::instance().reduceVelocity(0.5); // 50% speed reduction
                    log(WARNING, "Collision warning - velocity reduced");
                    break;
                    
                case CollisionRisk::NONE:
                    // Normal operation
                    break;
            }
            
            // Maintain monitoring frequency
            auto endTime = std::chrono::high_resolution_clock::now();
            auto elapsed = std::chrono::duration_cast<std::chrono::milliseconds>(endTime - startTime);
            
            if (elapsed < checkInterval_) {
                std::this_thread::sleep_for(checkInterval_ - elapsed);
            } else {
                log(WARNING, "Collision detection cycle took " + 
                    std::to_string(elapsed.count()) + "ms (target: " + 
                    std::to_string(checkInterval_.count()) + "ms)");
            }
        }
    }
};
\end{lstlisting}

\subsection{Reliability Engineering}
\label{subsec:reliability_engineering}

The system implements comprehensive reliability measures to ensure consistent operation and minimal downtime.

\subsubsection{Fault Tolerance Mechanisms}

\begin{lstlisting}[language=C++, caption=Redundant System Architecture]
class RedundantSystemManager {
private:
    struct ComponentStatus {
        bool isActive;
        bool isFaulty;
        std::chrono::steady_clock::time_point lastHealthCheck;
        int failureCount;
        ComponentHealth health;
    };
    
    enum class ComponentType {
        SENSOR_ENCODER,
        SENSOR_FORCE,
        ACTUATOR_MOTOR,
        CONTROLLER_MAIN,
        CONTROLLER_SAFETY,
        COMMUNICATION_PRIMARY,
        COMMUNICATION_BACKUP
    };
    
    std::unordered_map<ComponentType, std::vector<ComponentStatus>> components_;
    std::mutex componentMutex_;
    
    // Health monitoring
    std::thread healthMonitorThread_;
    std::atomic<bool> monitoringActive_{true};
    
public:
    RedundantSystemManager() {
        initializeRedundantComponents();
        healthMonitorThread_ = std::thread(&RedundantSystemManager::monitorHealth, this);
    }
    
    template<typename T>
    std::optional<T> getRedundantReading(ComponentType type,
                                        std::function<T(int)> readFunction) {
        std::lock_guard<std::mutex> lock(componentMutex_);
        
        auto& componentList = components_[type];
        std::vector<T> readings;
        std::vector<int> validIndices;
        
        // Collect readings from all active components
        for (size_t i = 0; i < componentList.size(); ++i) {
            if (componentList[i].isActive && !componentList[i].isFaulty) {
                try {
                    T reading = readFunction(static_cast<int>(i));
                    readings.push_back(reading);
                    validIndices.push_back(static_cast<int>(i));
                } catch (const std::exception& e) {
                    // Mark component as faulty
                    componentList[i].isFaulty = true;
                    componentList[i].failureCount++;
                    log(ERROR, "Component failure in redundant reading: " + std::string(e.what()));
                }
            }
        }
        
        if (readings.empty()) {
            log(ERROR, "No valid readings available for component type " + 
                std::to_string(static_cast<int>(type)));
            return std::nullopt;
        }
        
        // Use voting algorithm for consensus
        return performVoting(readings, validIndices);
    }
    
    bool switchToBackup(ComponentType type, int primaryIndex) {
        std::lock_guard<std::mutex> lock(componentMutex_);
        
        auto& componentList = components_[type];
        
        // Mark primary as faulty
        if (primaryIndex < componentList.size()) {
            componentList[primaryIndex].isActive = false;
            componentList[primaryIndex].isFaulty = true;
            componentList[primaryIndex].failureCount++;
        }
        
        // Find available backup
        for (size_t i = 0; i < componentList.size(); ++i) {
            if (i != primaryIndex && !componentList[i].isFaulty && !componentList[i].isActive) {
                componentList[i].isActive = true;
                log(INFO, "Switched to backup component " + std::to_string(i) + 
                    " for type " + std::to_string(static_cast<int>(type)));
                return true;
            }
        }
        
        log(ERROR, "No backup available for component type " + 
            std::to_string(static_cast<int>(type)));
        return false;
    }
    
private:
    template<typename T>
    std::optional<T> performVoting(const std::vector<T>& readings, 
                                  const std::vector<int>& indices) {
        if (readings.size() == 1) {
            return readings[0];
        }
        
        // For numerical values, use median voting
        if constexpr (std::is_arithmetic_v<T>) {
            std::vector<T> sortedReadings = readings;
            std::sort(sortedReadings.begin(), sortedReadings.end());
            
            size_t middle = sortedReadings.size() / 2;
            if (sortedReadings.size() % 2 == 0) {
                return (sortedReadings[middle - 1] + sortedReadings[middle]) / 2;
            } else {
                return sortedReadings[middle];
            }
        }
        
        // For other types, use majority voting or first valid reading
        return readings[0];
    }
    
    void monitorHealth() {
        while (monitoringActive_) {
            {
                std::lock_guard<std::mutex> lock(componentMutex_);
                
                for (auto& [type, componentList] : components_) {
                    for (size_t i = 0; i < componentList.size(); ++i) {
                        if (componentList[i].isActive) {
                            // Perform health check
                            ComponentHealth health = performHealthCheck(type, i);
                            componentList[i].health = health;
                            componentList[i].lastHealthCheck = std::chrono::steady_clock::now();
                            
                            // Check for degradation
                            if (health.status == HealthStatus::DEGRADED) {
                                log(WARNING, "Component degradation detected: type=" + 
                                    std::to_string(static_cast<int>(type)) + ", index=" + std::to_string(i));
                                
                                // Consider switching to backup if available
                                if (health.reliability < 0.7) { // Less than 70% reliability
                                    switchToBackup(type, i);
                                }
                            }
                        }
                    }
                }
            }
            
            std::this_thread::sleep_for(std::chrono::seconds(1));
        }
    }
    
    ComponentHealth performHealthCheck(ComponentType type, size_t index) {
        ComponentHealth health;
        
        switch (type) {
            case ComponentType::SENSOR_ENCODER:
                health = checkEncoderHealth(index);
                break;
            case ComponentType::SENSOR_FORCE:
                health = checkForceSensorHealth(index);
                break;
            case ComponentType::ACTUATOR_MOTOR:
                health = checkMotorHealth(index);
                break;
            default:
                health.status = HealthStatus::UNKNOWN;
                health.reliability = 0.5;
                break;
        }
        
        return health;
    }
};
\end{lstlisting}

\subsection{System Diagnostics and Monitoring}
\label{subsec:diagnostics_monitoring}

Comprehensive diagnostics enable proactive maintenance and early problem detection.

\subsubsection{Built-in Test Equipment (BITE)}

\begin{lstlisting}[language=C++, caption=Self-Diagnostic System]
class SelfDiagnosticSystem {
public:
    enum class DiagnosticLevel {
        POWER_ON_SELF_TEST,    // Basic functionality check
        PERIODIC_BUILT_IN_TEST, // Regular health monitoring
        INITIATED_BUILT_IN_TEST, // On-demand comprehensive test
        CONTINUOUS_MONITORING   // Real-time system monitoring
    };
    
    struct DiagnosticResult {
        std::string testName;
        bool passed;
        double confidence;
        std::string details;
        std::chrono::steady_clock::time_point timestamp;
        std::vector<std::string> recommendations;
    };
    
private:
    std::vector<std::unique_ptr<DiagnosticTest>> tests_;
    std::unordered_map<std::string, DiagnosticResult> lastResults_;
    std::mutex resultsMutex_;
    
    // Continuous monitoring
    std::thread monitoringThread_;
    std::atomic<bool> monitoringActive_{true};
    
public:
    SelfDiagnosticSystem() {
        initializeDiagnosticTests();
        monitoringThread_ = std::thread(&SelfDiagnosticSystem::continuousMonitoring, this);
    }
    
    std::vector<DiagnosticResult> runDiagnostics(DiagnosticLevel level) {
        std::vector<DiagnosticResult> results;
        
        for (const auto& test : tests_) {
            if (test->getLevel() <= level) {
                DiagnosticResult result = test->execute();
                results.push_back(result);
                
                // Store result for historical tracking
                {
                    std::lock_guard<std::mutex> lock(resultsMutex_);
                    lastResults_[result.testName] = result;
                }
                
                // Take immediate action if test failed
                if (!result.passed && test->isCritical()) {
                    handleCriticalFailure(result);
                }
            }
        }
        
        return results;
    }
    
    DiagnosticSummary generateHealthReport() {
        std::lock_guard<std::mutex> lock(resultsMutex_);
        
        DiagnosticSummary summary;
        summary.overallHealth = calculateOverallHealth();
        summary.criticalIssues = identifyCriticalIssues();
        summary.warnings = identifyWarnings();
        summary.recommendations = generateRecommendations();
        summary.timestamp = std::chrono::steady_clock::now();
        
        return summary;
    }
    
private:
    void initializeDiagnosticTests() {
        // Hardware tests
        tests_.push_back(std::make_unique<MotorDiagnosticTest>());
        tests_.push_back(std::make_unique<SensorDiagnosticTest>());
        tests_.push_back(std::make_unique<CommunicationDiagnosticTest>());
        tests_.push_back(std::make_unique<PowerSystemDiagnosticTest>());
        
        // Software tests
        tests_.push_back(std::make_unique<MemoryDiagnosticTest>());
        tests_.push_back(std::make_unique<TimingDiagnosticTest>());
        tests_.push_back(std::make_unique<CalibrationDiagnosticTest>());
        
        // Safety tests
        tests_.push_back(std::make_unique<EmergencyStopTest>());
        tests_.push_back(std::make_unique<CollisionDetectionTest>());
        tests_.push_back(std::make_unique<ForceLimitTest>());
    }
    
    void continuousMonitoring() {
        while (monitoringActive_) {
            // Run periodic diagnostics
            auto results = runDiagnostics(DiagnosticLevel::CONTINUOUS_MONITORING);
            
            // Analyze trends and predict failures
            analyzeTrends();
            
            // Update system health indicators
            updateHealthIndicators();
            
            std::this_thread::sleep_for(std::chrono::seconds(5));
        }
    }
    
    double calculateOverallHealth() {
        double totalWeight = 0.0;
        double weightedScore = 0.0;
        
        for (const auto& [testName, result] : lastResults_) {
            double weight = getTestWeight(testName);
            double score = result.passed ? result.confidence : 0.0;
            
            totalWeight += weight;
            weightedScore += weight * score;
        }
        
        return totalWeight > 0 ? weightedScore / totalWeight : 0.0;
    }
    
    void handleCriticalFailure(const DiagnosticResult& result) {
        log(CRITICAL, "Critical diagnostic failure: " + result.testName + 
            " - " + result.details);
        
        // Trigger appropriate safety response
        if (result.testName.find("Emergency") != std::string::npos) {
            EmergencyStopSystem::instance().triggerEmergencyStop(
                EmergencyStopSystem::StopReason::SYSTEM_FAULT,
                "Diagnostic failure: " + result.testName);
        } else if (result.testName.find("Safety") != std::string::npos) {
            SystemSafetyMonitor::instance().activateSafeMode();
        }
    }
};

// Example diagnostic test implementation
class MotorDiagnosticTest : public DiagnosticTest {
public:
    DiagnosticResult execute() override {
        DiagnosticResult result;
        result.testName = "Motor Health Check";
        result.timestamp = std::chrono::steady_clock::now();
        
        try {
            // Test motor response
            double responseTime = testMotorResponse();
            double currentDraw = measureCurrentDraw();
            double temperature = measureMotorTemperature();
            double vibration = measureVibrationLevel();
            
            // Analyze results
            bool responseOk = responseTime < 50.0; // ms
            bool currentOk = currentDraw < getMaxCurrentLimit();
            bool temperatureOk = temperature < getMaxTemperature();
            bool vibrationOk = vibration < getMaxVibrationLevel();
            
            result.passed = responseOk && currentOk && temperatureOk && vibrationOk;
            result.confidence = calculateConfidence(responseTime, currentDraw, 
                                                  temperature, vibration);
            
            // Generate detailed report
            std::stringstream details;
            details << "Response time: " << responseTime << "ms, ";
            details << "Current draw: " << currentDraw << "A, ";
            details << "Temperature: " << temperature << "$^\\circ$C, ";
            details << "Vibration: " << vibration << "g";
            result.details = details.str();
            
            // Generate recommendations if needed
            if (!result.passed) {
                generateMaintenanceRecommendations(result, responseTime, 
                                                 currentDraw, temperature, vibration);
            }
            
        } catch (const std::exception& e) {
            result.passed = false;
            result.confidence = 0.0;
            result.details = "Test execution failed: " + std::string(e.what());
        }
        
        return result;
    }
    
    DiagnosticLevel getLevel() const override {
        return DiagnosticLevel::PERIODIC_BUILT_IN_TEST;
    }
    
    bool isCritical() const override {
        return true; // Motor failures are critical for robotic system
    }
};
\end{lstlisting}

This comprehensive safety and reliability analysis demonstrates the system's robust approach to ensuring patient safety and operational dependability through multiple layers of protection, continuous monitoring, and proactive fault detection and mitigation strategies.
